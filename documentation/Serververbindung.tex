% This file was converted from HTML to LaTeX with
% gnuhtml2latex program
% (c) Tomasz Wegrzanowski <maniek@beer.com> 1999
% (c) Gunnar Wolf <gwolf@gwolf.org> 2005-2010
% Version : 0.4.
\section{Serververbindung}
Ein paar Informationen zur Serververbindung bzw. Sockets in Pyhton:\\
Sockets in Python funktionieren ähnlich wie in Java. Per \grqq import socket\grqq{} stehen sie uns zur Verfügung. s = socket.socket(socket.AF\_INET, socket.SOCK\_STREAM) legt einen TCP Socket mit dem IPv4 Protokol an. Mittels s.connect(ip,port) verbindet man sich mit dem Server. Anschließend kann in einer while-Schleife auf Servernachrichten gewartet werden.\\
\\
Agent und Server kommunizieren über das TCP Protokoll über den vordefinierten Standardport 3100. Bevor ein Agent am Spiel teilnehmen kann, muss er sich \grqq registrieren\grqq{}. Das geschieht über zwei Kommandos.\\
CreateEffector message: (scene $<$filename$>$). Dieser Befehl legt die .rsg Datei fest welche der Server für diesen Agent nutzt (enthält physische Repräsentation und effectors/perceptors).\\
InitEffector message: (init (unum $<$playernumber$>$)(teamname $<$yourteamname$>$)). Hier gibt der Agent an, zu welchem Team er gehört und welche Spielernummer er hat. Als Spielernummer ist auch 0 möglich - hier weist der Server dem Agent die nächste freie Nummer zu. Sofern wir keine festen Rollen verteilen, dürfte die Spielernummer aber irrelevant sein.\\
Sofern ich nichts übersehen habe, gibt es keine Rückmeldung vom Server - es gibt also keine Rückmeldung für gesendete Kommandos.\\
\\
Die Frage wäre nun, wie Komplex das Serververbindungs-Modul sein darf/soll. Stupides Management der Verbindung (aufbauen, aufrechterhalten, senden, empfangen). Also reines Senden und Empfangen von Nachrichten ohne große Verifikationen (außer natürlich sicherzustellen, dass alles vollständig und fehlerfrei gesendet wird/ankommt). Oder darf es doch mehr wie z.B. das Registrieren mit dem Server übernehmen oder nach bestimmten Richtlinien Nachrichten validieren.\\
\\
Quellen:\\
\url{http://simspark.sourceforge.net/wiki/index.php/Network\_Protocol}\\
\url{http://simspark.sourceforge.net/wiki/index.php/Effectors}\\
\url{http://openbook.galileocomputing.de/python/python\_kapitel\_20\_001.htm}\\