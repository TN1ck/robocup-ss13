% This file was converted from HTML to LaTeX with
% gnuhtml2latex program
% (c) Tomasz Wegrzanowski <maniek@beer.com> 1999
% (c) Gunnar Wolf <gwolf@gwolf.org> 2005-2010
% Version : 0.4.
\section{Fähigkeiten}
\subsection{Allgemeines}
Bevor angefangen wurde, Bewegungen für die viertuellen NAOs zu 
programmieren, wurde überlegt, welche Fähigkeiten in der 3d-Simulation 
benötigt werden könnten. Die Bewegungn wurden im vorhinein sortiert, 
danach ob sie für uns im ersten Moment umsetzbar klangen oder eher 
nicht. Auch wurde sich Gedanken darüber gemacht, was der Torwart für 
zusätzliche Bewegungen braucht. Im Folgen sind die wesentlichen 
Bewegungen aufgezählt und kurz beschrieben.

\subsubsection{Relevante (allg.) Bewegungen}

\begin{tabular}{|c|p{12cm}|}
\hline 
Bewegung & Kurzbeschreibung und Anwendung \\ 
\hline 
Laufen & Die Laufbewegung wird nicht implementiert, da der NAO auf Grund von sonst aufwändigen Algorithmen über das Feld gebeamt wird. \\ 
\hline 
Schuss & Soll per Tritt oder per "dagegen Beamen" realisiert werden, das wird 
u.U. testen. Der Schuss ist relevant für den Angriff und für Elfmeter. \\ 
\hline 
Pass & Der Pass ist der gerichtete Schuss zu einem eigenen Mitspieler oder 
zu einer bestimmten Position. Er wird vermutlich auch (größtenteils) für
 "Einwurf", Anstoß, Abstoß, Ecke und Freistoß verwendet. \\ 
\hline 
Kopfbewegung & Der NAO muss sich umsehen, um seine Umgebung wahrnehmen zu können. 
Die Kopfbewegung ist also dringend nötig. Es wird wahrscheinlich ein 
Kopfbewegung zum rund um gucken und eine um an eine bestimmte Position 
zu sehen geben. \\ 
\hline 
Armbewegung & Die Armbewegung ist u.U. relevant für die Kommunikation in gewissem Maße. Auch kann sie für den Torwart von nutzen sein. \\ 
\hline 
"Dribblen" & Soll eine Kombination aus Schießen (Passen) und Laufen sein. \\ 
\hline 
Aufstehen & Nach dem Hinfallen oder Umkippen muss der NAO wieder aufstehen 
können. Dieses soll nicht durch einfaches Beamen realisiert werden, 
falls das möglich sein sollte. (evtl. wird der NAO in der gleichen 
Position gebeamt, was bedeutet er liegt immer noch) \\ 
\hline 
Drehen & Um sich umzusehen oder die Laufrichtung zu ändern sehr relevant. Soll auch durch Beamen passieren.\\ 
\hline 
\end{tabular} 

\subsubsection{Torwartspezifische Bewegungen}

\begin{tabular}{|c|p{11cm}|}
\hline 
Bewegung & Kurzbeschreibung und Anwendung \\ 
\hline 
Halten & Der Keeper soll versuchen den Ball nicht ins Tor zu lassen. Dafür 
benötigt er gewisse Bewegungen, z.B. dem Ball entgegenlaufen, Arm 
ausstrecken, Springen. \\ 
\hline 
Zum Ball springen & Bewegung des Keepers, die den Torwart möglichst viel verdecken lässt vom Tor, damit der Ball nicht hineingeschossen werden kann. \\ 
\hline 
Aufstehen & Evtl. ist es nötig eine extra Aufstehbewegung zu kreieren, falls der 
Ball zu gefährlich liegt und durch das Aufstehen ins Tor beförtert wird. \\ 
\hline 
\end{tabular} 

\subsubsection{Irrelevante Bewegungen}

\begin{tabular}{|c|p{12cm}|}
\hline 
Bewegung & Kurzbeschreibung und Anwendung \\ 
\hline 
Fallrückzieher & Reaktionstechnisch (und möglicherweise auch durch die Technik des 
NAO's) vermutlich nicht realisierbar. Definitiv aber erstmal unnötig für
 uns. \\ 
\hline 
Kopfball & Vermutlich aufgrund gewisser Reaktionszeiten nicht machbar. \\ 
\hline 
\end{tabular} 

\subsubsection{Bewegen der Gelenke}
Für die Kommunikation mit dem Server werden zwei verschiedene 
Bezeichnung genutzt. Die Perceptor names sind die Namen, die der Server 
sendet, damit der NAO weiß, wie seine Gelenke stehen. Die Effector names
 werden mit den zugehörigen Geschwindigkeitenan den Server geschickt. 
Die Geschwindigkeiten geben an wie schnell sich die Gelenke ab dem 
nächsten Zyklus bewegen sollen. Dabei ist zu beachten, dass die Gelenke 
eine maximale Auslenkung besitzen.\\
Die Gelenke werden prinzipiell durch den folgenden Befehl bewegt:\\
\textbf{send(socket, message)}\\
Die \textit{message} sieht dabei wie folgt aus:\\
\textbf{message =} (EffectorName  AngularSpeed)\\
Die Perceptor und Effector names sind in der folgenden \href{http://simspark.sourceforge.net/wiki/index.php/Models}{Tabelle} aufgeführt.

\begin{tabular}{|c|c|c|c|}
\hline 
No. & Desc. &  Perceptor name & Effector name \\ 
\hline 
1 & Neck Yaw & hj1 & he1 \\ 
\hline 
2 & Neck Pitch & hj2 & he2 \\ 
\hline 
3 & Left Shoulder Ptch & laj1 & lae1 \\ 
\hline 
4 & Left Shoulder Yaw & laj2 & lae2\\ 
\hline 
5 & Left Arm Roll & laj3 & lae3 \\ 
\hline 
6 & Left Arm Yaw & laj4 & lae4 \\ 
\hline 
7 & Left Hip YawPitch & llj1 & lle1 \\ 
\hline 
8 & Left Hip Roll & llj2 & lle2 \\ 
\hline 
9 & Left Hip Pitch & llj3 & lle3 \\ 
\hline 
10 & Left Knee Pitch & llj4 & lle4 \\ 
\hline 
11 & Left Foot Pitch & llj5 & lle5 \\ 
\hline 
12 & Left Foot Roll & llj6 & lle6 \\ 
\hline 
13 & Right Hip YawPitch & rlj1 & rle1 \\ 
\hline 
14 & Right Hip Roll & rlj2 & rle2 \\ 
\hline 
15 & Right Hip Pitch & rlj3 & rle3 \\ 
\hline 
16 & Right Knee Pitch & rlj4 & rle4 \\ 
\hline 
17 & Right Foot Pitch & rlj5 & rle5 \\ 
\hline 
18 & Right Foot Roll & rlj6 & rle6 \\ 
\hline 
19 & Right Shoulder Pitch & raj1 & rae1 \\ 
\hline 
20 & Right Shoulder Yaw & raj2 & rae2 \\ 
\hline 
21 & Right Arm Roll & raj3 & rae3 \\ 
\hline 
22 & Right Arm Yaw & raj4 & rae4 \\ 
\hline 
\end{tabular} 

\subsection{Umsetzung}
Nachdem die Bewegungen grob aufgelistet waren, wurde überlegt, welche
 Möglichkeit der Umsetzung es gibt. Für viele Bewegungen ist es möglich 
einen Keyframe zu erzeugen und diese in einer Keyframe-Engine 
auszuwerten, da diese nicht von Parametern oder nicht direkt von äußeren
 Einwirkungen abhängen. Für andere müssen andere Algorithmen geschrieben
 werden.

\subsubsection{Keyframes}
Ein Keyframe ist bei uns eine Abfolge von Momentaufnahmen von den 
Stellungen der Gelenke des NAOs. Der NAO hat 22 Gelenke, dessen 
Auslenkungswinkel jeweils angegeben wird. Diese einzelnen Positionen der
 Gelenke speichern wir in einem mehrdimensionalen Array ([n][23], wobei n
 die Anzahl der Frames für die jeweilige Bewegung ist). An den Stellen 
[i][0] für 
 steht die Zeit, wie lange der einzelne Frame dauern soll. Dann Folgen 
die Gelenke, die so sortiert sind, wie die Reihenfolge im Array "name", 
in dem die Namen der Gelenke stehen. Das war nötig, da wir verschiedene 
Reihenfolgen der Gelenkbenennenung hatten und so eine Zuordnung möglich 
ist.\\
Die Keyframes erschaffen in unserem Projekt definierte 
Bewegungen, die häufiger ausgeführt werden müssen.
Damit wir nicht alle Keyframes selbst erzeugen müssen, wollten wir die 
Aufstehbewegungen (vom Bauch und vom Rücken) der \href{http://www.naoteamhumboldt.de/de/}{Humboldt Universität}
 benutzen, die uns diese zur Verfügung gestellt hat.
Das Aufstehen vom Rücken konnten wir auch mit einigen veränderungen 
benutzen, jedoch wollte unser NAO mit unserer Keyframe-Engine nicht aus 
seiner Bauchlage aufstehen, weshalb wir diesen neu entwickelten.
Momentan stehen folgende Bewegungen durch Keyframes zur Verfügung:

\begin{itemize}
\item  lookAround (umsehen)
\item  stand\_up\_from\_back
\item  stand\_up\_from\_front
\item  kick1 (ein erster sanfter Schuss)
\end{itemize}
\subsubsection{Berechnungen (Keyframe-Engine)}
Die Berechnunung findet in unserer Keyframe-Engine statt. Dort stehen die einzelnen Bewegungs-Methoden zur Verfügung.

\begin{itemize}
\item  lookAround()
\item  stand\_up\_from\_back()
\item  stand\_up\_from\_front()
\item  kick1()
\end{itemize}
Alle diese Methoden holen sich als erstes das jeweilige 
Keyframe-Array und die Reihenfolge der Gelenke mit dem Array name. Da 
die Keyframe-Engine in jedem Zyklus aufgerufen wird, muss (der aktuelle 
Keyframe,) die aktuelle Zeile und die vergangene Zeit des Frames 
gespeichert werden. Zusätlich überprüfen wir auch, ob wir im letzten 
Frame der Bewegung angekommen sind. Sind wir im letzten Frame, werden 
alle Gelenkgeschwindigkeiten auf 0.0 gesetzt, damit sie sich nicht 
weiter bewegen.
Die eigentliche Berechnung, der nächsten Geschwindigkeiten findet
 in der Methode get\_new\_joint\_postion(keyframe, name). Dort werden die 
jeweiligen Gelenkepaare, bestehen aus aktueller Gelenkposition und der 
Position, wo wir in dem aktuellen Frame hinwollen berechnet. Dabei 
müssen auf mehrere Sachen geachtet werden:

\begin{enumerate}
\item  Die aktuelle Gelenkposition ist die vom Zyklus davor, d.h. wir haben einen Versatz von 20ms.
\item  Wir wollen nicht im nächsten Zyklus die gesamte Bewegung ausführen, sondern nur die, die in den nächsten 20ms passieren soll.
\item  Das Gelenk darf nicht über seine maximale Auslenkung gesteuert werden.
\end{enumerate}
Jede Gelenkgeschwindigkeit wird dann mit der folgenden Formel berechnet:
\begin{equation}
\texttt{Winkel} = \frac{20ms \cdot (\overbrace{keyframe\_ joint}^{\texttt{Winkel aus Keyframe}} - (\overbrace{hinge\_ joint}^{\texttt{akt. Gelenkstellung}} + \overbrace{last\_ joint\_ angle}^{\texttt{letzter berechneter Winkel}}))}{\underbrace{keyframe[0]}_{\texttt{Dauer akt. Frames}} - \underbrace{progressed\_ time}_{\texttt{vergangene Zeit akt. Frames}}}
\end{equation}

Dieser Winkel wird für den nächsten Zyklus gespeichert und in Rad/sec umgerechnet.
\begin{equation}
\texttt{Winkelgeschwindigkeit} = \frac{\overbrace{rad(Winkel)}^{\texttt{oben berechneter Winkel}}}{\underbrace{20 * 0,001}_{\texttt{20ms}}}
\end{equation}
Danach wird die abgelaufe Zeit um 20ms erhöht und wenn die Zeile 
des Keyframes abgearbeitet ist wird die Zeile um 1 erhöht. Ist die 
Bewegung am Ende angekommen, wird die aktuelle Zeile aud 0 gesetzt.

\subsubsection{Einzelne Bewegungen}
Im Folgenden werden die einzelnen Bewegungen vorgestellt und dessen Umsetztung beschrieben.

\paragraph{Aufstehbewegungen}
Das Aufstehen ist besonders gut geeignet mit einem Keyframe umgesetzt
 zu werden, da hier keine Parameter nötig sind. Da es für den NAO einen 
Unterschied macht, ob er auf dem Rücken liegt oder auf dem Bauch mussten
 zwei verschiedene Keyframes entwickelt werden.

\subparagraph{Aufstehen vom Rücken aus}
Wurde als erstes voll funktionstüchtig implementiert. Sobald der NAO 
auf dem Rücken liegt, steht er wieder auf und stellt sich nahezu 
vollständig gerade wieder hin. Dieser Keyframe wurde von der \href{http://www.naoteamhumboldt.de/de/}{Humboldt Universität} benutzt und musste nur an einigen Stellen verändert werden.

\subparagraph{Aufstehen vom Bauch aus}
Hier hatten wir einen Keyframe von der \href{http://www.naoteamhumboldt.de/de/}{Humboldt Universität} zur Verfügung, der unseren NAO jedoch nicht aufstehen lassen hat. Deshalb wurde er entwickelt. ...

\paragraph{Kopfbewegungen}
Die Bewegung des Kopfes ist nicht nur wichtig, um sich umzusehen, 
damit der NAO z.B. seine Position besser bestimmen kann oder den Ball 
sehen kann, wenn er außerhalb des Sichtfeldes seines ist, sondern auch 
zur Fixierung eines Punktes.

\subparagraph{Umsehen }
Dies ist eine feste Abfolge von Bewegungen. Es soll eine möglichst 
großer Bereich gesehen werden. Dafür wird ein Keyframe geschrieben, da 
hier keine Berechnungen gebraucht werden. Dazu wird der Kopf zuerst nach
 linkt, nach unten  und dann nach rechts bewegt.

\subparagraph{Auf eine Punkt gucken}
Diese Bewegung benötigt eine Berechnung, die ausgehend von der 
aktuellen Stellung der Kopfgelenke den Weg zum gewünschten Punkt 
berechnet. Es ist noch nicht geklärt, mit welchen Daten die Berechnung 
stattfinden sollen. Dazu ist eine Absprache vor allem mit der 
Taktikgruppe nötig.

\paragraph{Schuss}
Für den Schuss sind bis jetzt Keyframes geplant. Dabei soll 
unterschieden werden, ob der Schuss weit oder kurz sein soll (bis jetzt 2
 Keyframes). Schwierigkeiten, die wir dabei bekommen könnten, sind dass 
der NAO ziemlich genau vor dem Ball stehen muss und dass er das 
Gleichgewicht halten muss.

\subsection{Interfaces}
\subsubsection{Variablen}
\textbf{Working}
Wenn diese Variable auf True gesetzt ist, ist ein Keyframe noch nicht 
beendet. Daher darf kein Movement Befehl ausgeführt werden und nur 
Keyframes gestartet werden, die den Kopf bewegen

\subsubsection{Funktionen}
\textbf{work}
Diese Funktion muss in jedem Cycle aufgerufen werden. Sie überprüft ob 
aktuell noch Keyframes auszuführen sind, ist dies der Fall werden die 
Geschwindigkeiten für diesen Cycle berechnet.

\subsubsection{Keyframe Aufrufe}
Diese Funktionen werden einmal aufgerufen und bestimmen damit den 
aktuell auszuführenden Keyframe. Die Bewegung des Naos erfolgt 
ausschließlich über \textbf{work}.

\begin{tabular}{|c|p{9cm}|}
\hline 
Name & Funktion \\ 
\hline 
stand() & Grundhaltung des Nao, die nach dem Hinzufügen eines Nao aufgerufen 
werden sollte, um aus der vordefinierten Haltung in einen normalen Stand
 zu kommen. \\ 
\hline 
stand\_up\_from\_back() & Diese Bewegung lässt den Nao aufstehen, wenn er auf dem Rücken liegt (modifizierte Humboldt-Version). \\ 
\hline 
stand\_up\_from\_front() & Diese Bewegung lässt den Nao aufstehen, wenn er auf dem Bauch liegt 
(eigenständig geschriebener Keyframe der sich an der Aufstehbewegung der
 zweitplatzierten Mannschaft, der Osaka Open 2011 orientiert). \\ 
\hline 
kick\_left() / kick\_right() & Dieser Tritt ist ein sehr kurzer, erster Versuch einen Tritt auszuführen. Er könnte in Zukunft zum Dribbling verwendet werden. \\ 
\hline 
kick\_strong\_left() / kick\_strong\_right() & Ein erster starker Tritt wird insbesondere für Torschüsse benötigt. 
Er lässt den Nao regelmäßig umkippen, bei einem solchen Schuss lässt 
sich damit aber arbeiten. \\ 
\hline
kick\_in\_left() / kick\_in\_right() & Eine starke Schussvariante, die insbesondere für Anstoß, Abstoß, Freistoß, Eckstoß und Strafstoß gedacht ist. \\ 
\hline 
head\_lookAround() & Wird eine vollständige Bewegung zum umschauen benötigt, so wird \textbf{head\_lookAround} verwendet. \\ 
\hline 
head\_move(angle) & Bewegt den Kopf horizontal um den angegebenen Winkel. Positive Werte 
bewegen den Kopf, vom Nao aus gesehen, nach links; negative nach rechts.
 Maximalwerte liegen bei 120 bzw -120 Grad. Die Bewegung wird vertikal 
betrachtet auf der 0 Grad Ebene ausgeführt. \\ 
\hline 
head\_stop() & Stoppt die Kopfbewegung an der aktuellen Position. Aufgrund des 
Delays stoppt der Kopf allerdings einen Schritt weiter als an der im 
aktuellen Cycle von den Sensoren angegebene Position. \\ 
\hline 
head\_reset() & Setzt den Kopf auf 0 Grad horizontal und vertikal zurück. \\ 
\hline 
head\_down() & Bringt den Nao dazu, nach unten zu schauen. \\ 
\hline 
fall\_on\_front() & Diese Testfunktion lässt den Nao auf den Bauch fallen. \\ 
\hline 
fall\_on\_back() & Diese Testfunktion lässt den Nao auf den Rücken fallen. \\ 
\hline 
parry\_left() & Der Torwart führt eine Parade nach links aus und steht nach kurzer Zeit wieder auf. \\ 
\hline 
parry\_right() & Der Torwart führt eine Parade nach rechts aus und steht nach kurzer Zeit wieder auf. \\ 
\hline 
parry\_straight() & Der Torwart versucht möglichst viel Weg zu versperren wenn der Ball nahezu gerade auf ihn zukommt. \\ 
\hline 
parry\_straight1() & Der Torwart geht in die Hocke um den Ball abzufangen wenn der Ball exakt auf ihn zukommt. \\ 
\hline 
\end{tabular} 
