% This file was converted from HTML to LaTeX with
% gnuhtml2latex program
% (c) Tomasz Wegrzanowski <maniek@beer.com> 1999
% (c) Gunnar Wolf <gwolf@gwolf.org> 2005-2010
% Version : 0.4.
\section{Kommunikation}
\subsection{Grundlegendes}
Für die Kommunikation zwischen den NAOs stellt der Server 2 
Schnittstellen zur Verfügung, den Say-Effector sowie den Hear-Perceptor.
 Diese erlauben es uns, einen NAO etwas sagen zu lassen und zu hören, 
was die anderen NAOs von sich geben. Die Kommunikation unterliegt 
allerdings einigen Einschränkungen:

\begin{itemize}
\item  Eine Nachricht ist maximal 20 Zeichen lang.
\item  Die Nachricht darf nur einen eingeschränkten (92) 
ASCII-Zeichensatz verwenden. (zur Zeit benutzen wir 90, um den Umgang 
mit Python zu erleichtern)
\item  Nachrichten können 50m weit gehört werden.
\item  Die NAOs können maximal alle 0.04s eine Nachricht hören, in 
der Zwischenzeit (also $<$0.04s nachdem sie etwas gehört haben) sind 
sie taub.
\item  Sprechen zwei NAOs gleichzeitig, wird nur eine Nachricht 
gehört. Der Sprecher hört jedoch immer seine eigene Nachricht. Die Teams
 sprechen versetzt, können sich also nicht blockieren.
\end{itemize}

\subsection{Konzept}
Die Nachrichten werden zurzeit nach dem folgenden Schema codiert und 
übertragen:
Zunächst wird eine Methode aufgerufen, welcher übergeben wird, was 
gesagt werden soll. 
Hierfür stehen sowohl Funktionen zur Verfügung, bei denen man über die 
Parameter genau definiert welche Nachricht gesendet werden soll;\\
\texttt{\%Beispielsweise sendet dieser Aufruf an NAO 5 die Nachricht, er soll zum Ball gehen\\
sayCommandTo(1, 5)}\\
als auch vorgefertigte Funktionen, welche die Codelesbarkeit und den Umgang mit der Kommunikation erleichtern sollen.\\
\texttt{\%Derselbe Aufruf wie oben.\\
sayGoToBall(5)}\\
\texttt{\%Fügt man weitere Parameter hinzu erweitert sich die Nachricht. Wir können z.B. auch Koordinaten des Balls mitliefern.\\
sayGoToBall(5, 8.52, -3.11)}\\
Die aufgerufene Methode codiert die Nachricht nun derart, dass 
sie für den Say-Effector brauchbar ist. Eine codierte Nachricht ist 
folgendermaßen aufgebaut:\\
$<$commandcode$><$sender$><$target$><$parameter1$>$...$<$parameterN$><$checksum$>$

\begin{itemize}
\item  commandcode - spezifiziert die Struktur und Aussage der Nachricht ("gehe zum Ball")
\item  target - enthält den Empfänger der Nachricht("5"). 0 steht hierbei für eine Nachricht an alle
\item  parameter1 bis N - enthalten mitgelieferte Daten wie Koordinaten oder dergleichen("8.52, -3.11")
\item  checksum - wird aus den übrigen Bestandteilen der Nachricht gebildet und validiert die Nachricht
\end{itemize}
Für das Beispiel von oben würden wir folgende Codierung erhalten: \$\{\&\{\%]NCS!GI\\
Nun wird diese codierte Nachricht über den Say-Effector an den Server geschickt.\\
Ein NAO hat nun über die Hear-Methode die Möglichkeit, diese 
Nachricht zu empfangen. Diese wird dann zunächst vom Translator 
decodiert. Daraus wird nun ein HearObject erstellt, dessen Variablen mit
 den übergebenen Parametern übereinstimmen. Dieses HearObject wird nun 
zurückgegeben, und kann jederzeit per eval() ausgewertet werden. 

\subsection{Commandcode - Tabelle}

\begin{tabular}{|c|c|c|}
\hline 
Commandcode & Message Type & Mögliche Parameter \\ 
\hline 
0 & reserved, do not use & - \\ 
\hline 
1 & goToBall & target(0), x(0.0), y(0.0) \\ 
\hline 
2 & standUp & target(0)\\ 
\hline 
3 & iAmHere & sender, x, y \\ 
\hline 
4 & youAreHere & target, x, y \\ 
\hline 
5 & ballPosition & x, y \\ 
\hline 
6..89 & customMessage & - \\ 
\hline 
\end{tabular}\\
In Klammern angegebene Werte sind Default-Werte, und müssen demnach nicht übergeben werden.
Die bisherigen Commandcodes stellen lediglich Beispiele dar. Weitere Messages einzufügen stellt kein Problem dar. 
Erwünschte Commandcodes, samt Namen und Parametern, einfach hier hinschreiben, und wir fügen sie so schnell es geht ein.