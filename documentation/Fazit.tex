\section{Fazit}

Ziel des Praktikums war es, ein Tor zu schießen und einen signifikante Unterschied auf ein Spiel gegen UT Austin Villa im Vergleich zu einem leeren Spielfeld zu haben.\\
Wir haben es geschafft uns über das Spielfeld zu bewegen, in dem wir uns beamen. Wir können uns auch lokalisieren indem wir feste Objekte benutzen. Diese Information und die daraus berechneten Positionen des Balles und der anderen Roboter werden in das Weltmodell des Agenten eingepflegt. Dabei wird die vergangene Zeit der Aktualisierung beachtet, damit nicht mit veralteten Informationen gearbeitet wird.\\
Die Agenten kommunizieren untereinander, um z.B. die Position des Balles an die Mitspieler zu senden. Wenn die NAOs umfallen bemerken sie das und können von allein wieder aufstehen.\\
Wir haben ein hybrides Verhalten zwischen Schwarm- und Regelverhalten. Das Schwarm verhalten sorgt dafür, dass die einzelnen Roboter nicht alle auf eine Stelle laufen, sondern versuchen einen bestimmten Abstand zueinander zu halten. Wenn ein NAO am Ball ist, dann beginnt er mit diesem auf des Tor das Gegeners zu laufen. Zum Schluss ist es uns gelungen ein Tor zu schießen.\\
Während der Implementierung haben sich folgende Probleme ergeben: Beim Beamen wird der Oberkörper des NAOs immer in eine aufrechte Position gebracht, wodurch er beim Fortbewegen nicht umfallen kann. Kolliediert er nun mit einem Gegener, fällt dieser um, wir jedoch nicht.\\
Auch benötigt der Agent sehr viel Zeit, um zu entscheiden, was er als nächstes tun wird.\\
Bei der Gruppenarbeit ergab sich ein Problem mit der Organisation. Wir haben oft Aufgaben nicht klar definiert und diese dann nicht direkt jemandem zugewiesen. Auch haben wir uns nicht an die erlernten Konzepte der MPGI3 Vorlesung gehalten, diese unter anderem vielen Stellen nicht anzuwenden, da wir es mit keinem System zu tun hatten, das einem Benutzer gegenüber gestellt wird.\\
Abschließend bleibt zu sagen, dass alle etwas dazu gelernt haben und die Ziele fast erreicht wurden.