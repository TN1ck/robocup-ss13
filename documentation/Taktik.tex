% This file was converted from HTML to LaTeX with
% gnuhtml2latex program
% (c) Tomasz Wegrzanowski <maniek@beer.com> 1999
% (c) Gunnar Wolf <gwolf@gwolf.org> 2005-2010
% Version : 0.4.
\section{Taktik}
\section{Generelle KI-Struktur}
Wir haben uns innerhalb der Taktik Gruppe für ein Schwarmverhalten entschieden.\\
Das heißt das jeder Nao einem spezifischen Set von simplen 
Verhaltensregeln folgt, die in ihrer Gesamtheit das komplexe Verhalten 
der Gruppe konstituieren.

\section{Beispiel}
Beispielsweise kann eine gleichmäßige Verteilung der Nao's auf dem Spielfeld durch Selbstorganisation
erreicht werden.\\
Jedem Agenten wird dabei vorgegeben innerhalb eines bestimmten 
Abstands zu seinen Mitspielern zu bleiben. Da alle Mitpsieler bestrebt 
sind diesen Abstand zu erreichen ergibt sich im laufe der Zeit eine 
gleichmäßige Verteilung der Spieler.\\
Dies passiert konkret in einer reaktiven Agentenstruktur durch Bewertungen des Nutzens (Utility) von bestimmten Handlungen.\\
In unserem Beispiel erhalten die Agenten 2 Algortihmen:

\begin{itemize}
\item Der erste Algorithmus,run\_to\_friend, gibt einen Utility Wert 
(Float) aus, der steigt wenn sich der Nao von seinem Mitspielern 
entfernt.
\item Der zweite Algorithmus,run\_away\_from\_friend, gibt einen Utility
 Wert aus, der steigt wenn sich der Nao seinem Mitspielern annährt.
\end{itemize}
Es wird immer die Handlung mit dem aktuell höchsten Utility Wert ausgeführt.\\
Entfernt sich nun der Nao zu weit von einem bestimmten Mitspilern
 in seiner Umgebung steigt der Nutzen der Handlung run\_to\_friend. 
Kommt er ihm zu nahe steigt der Nutzen von run\_away\_from\_friend.
Sind beide Utilities gleichgroß befindet sich der Nao im Gleichgewicht: 
der angestrebte Abstand zum anderen Mitspieler wurde erreicht.\\
Dies ist also eine Form von negativer Rückkopplung.\\
(Ein weiteres Beispiel für Selbstorganisation)

\section{Taktik-Utilities}
\begin{itemize}
\item  enemy\_owns\_ball()
\item  player\_owns\_ball(x)
\end{itemize}
\section{Taktik-Grundfunktionen}
\begin{itemize}
\item  run\_to\_ball()
\item  run\_to\_enemy\_goal()
\item  run\_to\_own\_goal()
\end{itemize}
\begin{itemize}
\item  run\_to\_enemy(x)
\item  run\_away\_from\_enemy(x)
\end{itemize}
\begin{itemize}
\item  stay()
\end{itemize}
\begin{itemize}
\item  run\_to\_friend(x)
\item  run\_away\_from\_friend(x)
\end{itemize}
\begin{itemize}
\item  run\_away\_from\_l1()
\item  run\_away\_from\_r2()
\end{itemize}
