\section{Einleitung}

Im Praktikum RoboCup 3D ging es darum Roboter in der Simulation Fußball spielen zu lassen. Dazu wurde die 3D-Simulationsumgebung simspark genutz. Dieses verwendet aktuell das Roboter Modell des NAOs. Dabei ist jeder Roboter auf dem Spielfeld ein eigenständiger Agent, der seine Umgebung wahrnehmen kann, sich bewegen kann und Entscheidungen trifft. Damit ist es möglich elf gegen elf Roboter nach Regeln gegeneinander spielen zu lassen.\\
Damit die NAOs Fußball spielen, muss erst ein wenig Vorarbeit geleistet werden. Vom Server bekommt jeder Agent übermittelt, was er sieht und Informationen über sich, wie z.B. seine Gelenkstellungen und seine Lage im Raum. Diese müssen dann ausgewertet werden. Mit Hilfe eines Modells der Welt ist es möglich die Position des NAOs zu bestimmten und zu wissen wo der Ball ist. Auch muss eine Möglichkeit gefunden werden, einzelne Körperteile an bestimmte Positionen zu bringen, damit der Roboter z.B. aufstehen kann. Da das Laufen, als sehr schwierig betrachtet wurde, wurde am Anfang des Praktikums entschieden dieses herauszulassen. Wir beamen den Agenten über das Feld, was auch wieder neue Schwierigkeiten, wie die Kollisionsvermeidung mit sich bringt. Zusätzlich ist es auch möglich zu hören, was die Spieler in der näheren Umgebung sagen und selbst was zu sagen. Wie Nachrichten übermittelt werden, war ein weiteres Thema, womit wir uns als Gruppe beschäftigt haben.\\
Die eben beschrieben Themengebiete habe wir auf Kleingruppen aufgeteilt, die diese bearbeitet haben. Diese Aufteilung haben wir in der Dokumentation beibehalten.